\documentclass[11pt,twoside]{article}
\input{6006-stuff}

\usepackage{enumitem}
\setlist[itemize]{leftmargin=*}


\usepackage[style=authoryear-comp, natbib=false, uniquename=false, 
hyperref=false, backend=biber, bibencoding=auto, backref=false, 
minnames=1,defernumbers=true, maxcitenames=2, maxbibnames=99]{biblatex}


% redefine authoryear-comp to get square bracket
\DeclareCiteCommand{\cite}
{\usebibmacro{cite:init}%
  \bibopenbracket
\usebibmacro{prenote}}
{\usebibmacro{citeindex}%
\usebibmacro{cite}}
{}
{\usebibmacro{postnote}%
\bibclosebracket}

\AtEveryBibitem{\clearfield{doi}}
\AtEveryBibitem{\clearfield{issn}}
\AtEveryBibitem{\clearfield{isbn}}



\addbibresource{references.bib}

\linespread{0.95}
\addtolength{\topskip}{-10mm}
\AtBeginDocument{\addtolength{\abovedisplayskip}{-1mm}}
\AtBeginDocument{\addtolength{\belowdisplayskip}{-1mm}}
\addtolength{\parindent}{-1mm}
\addtolength{\parskip}{-1mm}
\addtolength{\abovecaptionskip}{-2mm}
\addtolength{\belowcaptionskip}{-3mm}


\begin{document}

\handout{EE 529: Data Analytics in ECE,
Spring 2022}
%\handoutwithouttitle{EE 525X Course Information Handout}{Spring 2016}

\subsubsection*{Overview}
An introduction to data analytics covering topics relevant to electrical and 
computer
engineers.



\subsubsection*{Class information}

\begin{tabular}{llll}
Instructor:

& Aleksandar Dogand\v{z}i\'c & Coover 3119
        & \\
        & ald@iastate.edu & Office hours: M W 1--2 \\
        Lectures:
        & M W 4:25--5:45 &  Carver 0132 & \medskip \\

TA:

& Rahmat Adesunkanmi & \\
& rahma@iastate.edu &
Office hour: Th 12:30--1:30
\\
&
%https://iastate.webex.com/meet/rahma
& online


      \end{tabular}

\subsubsection*{Textbook}
We will mostly follow our course lecture notes and slides. Supplemental 
textbooks are \cite{Murphy21} (containing Python examples)
and
\cite{BlumHopcroftKannan2020}
(mathematically oriented). A good undergraduate-level book with relevant 
content on clustering and least-squares regression and classification is 
\cite{BoydVandenberghe2018}.  Also useful and free: 
\cite{HastieTibshiraniFriedman2009,Leskovec2019mining}.
Programming aspects (ISU free access, code on Github)
\cite{RaschkaMirjalili2022,MuellerGuido2016}. 

\subsubsection*{Prerequisites}
 Familiarity
with linear algebra and coding/programming. EE 322 or equivalent course in 
probability and random processes.
Familiarity with optimization will
be a plus. 




\subsubsection*{Syllabus (tentative)}

\begin{itemize}[noitemsep,nolistsep]
  \item Math basics, inner product, vectors, matrices,
  \item Modeling data in high dimensions,
  \item Regression: linear, logistic,
  \item Classification: nearest neighbors, SVMs, kernel methods, neural 
    nets,
  \item Visualization and dimensionality reduction: PCA, clustering, Isomap, 
    MDS,
  \item Graphs: Random walks, PageRank,
  \item Big Data: Compressed sensing and matrix completion,
  \item Introduction to streaming algorithms.
\end{itemize}

\subsubsection*{Homework}
Homework-problem sets will be handed out during the semester. Problem sets 
will involve a mix of theory and practical implementation. Theoretical 
problems may involve a short mathematical derivation,  an analysis of a 
particular data-processing technique, or a construction of a new method. 
Implementation problems will involve some degree of programming. Feel free 
to use a scientific programming environment that you are most comfortable 
with. Python/Matlab/R/Julia   will suffice for most problems.

\subsubsection*{Collaboration policy}
You are encouraged to collaborate on homework assignments. However, you 
must (a) {clearly acknowledge your collaborator}, and (b) compose your 
final writeup and/or code by yourself.  If two assignments are obviously 
identical to each other, then {both} will automatically receive a score of 
zero (0). Please talk to me in advance for any clarifications.

\subsubsection*{Course project}
Final exam consists of a course project. The goal of the project is open-ended; 
it can involve either: (a) conducting research on a specific topic of your 
choice, or (b) coding up a technique and testing it on a real-world dataset, or 
(c) both. The only requirement is that it should involve (any combination of) 
analysis, design, or implementation of a data analytics technique. 

Projects are conducted in groups of at most two (2). It will be particularly 
beneficial for you (and the rest of the class) if you can integrate the project 
with your own research interests. Start thinking of project ideas early and 
discuss them with me before finalizing. A two-page project proposal will be due 
on March 25. Project presentations will be carried out during Dead Week. A 
final report will be due on May 12. More details will be given out in the 
coming weeks.

\subsubsection*{Grading policy}
The final grade will be calculated on a score of 100 (homework:  60\%, project: 
40\%). No late submissions please!

\subsubsection*{Student Accommodations}
Meet with me if you have a documented disability and anticipate needing 
accommodations in this
course,  Please request that a
Disability Resources (DR) staff send a Student Academic Accommodation 
Request
(SAAR) form verifying your disability and specifying the accommodations you
need. DR is located in Room 1076 of Student Services. 


\printbibliography


\end{document}

